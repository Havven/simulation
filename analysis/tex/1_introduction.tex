\section{Introduction}

\subsection{Money and Cryptocurrencies}

There are three primary functions of money; to act as a unit of account, a medium of exchange
and as a store of value. In addition, money should ideally exhibit durability,
portability, divisibility, uniformity, limited supply, and acceptability.
Money has become almost invisible over the past few decades as payment technology has advanced.
Because of this, it is often lost upon users of money that it is itself a technology that can be
improved. Specifically, this means improving the performance of our six desirable properties. \\

\noindent Bitcoin as a technological improvement on existing forms of money is impressive because it manages to
simultaneously improve durability, portability, and divisibility.
Further, it does so without requiring the enforcement of a nation state from which to derive its value.
The Bitcoin supply is, therefore, not subject to control by any central authority.

This fixed monetary policy means that increased adoption has tended to drive the price up over time,
allowing Bitcoin to outperform other forms of money as a store of value, precisely because it is not
subject to debasement and devaluation. Unfortunately this fixed monetary supply creates the potential
for volatility in the short term because there is no mechanism within Bitcoin that can monitor or
adjust to changing demand for the currency.

Thus it has tended to be a poor medium of exchange and an even worse unit of account.
In order for something to perform well as a medium of exchange or unit of account it must remain
relatively stable against other goods and services, because money is ultimately a good that other goods
are denominated in. If the price of money as a good is too varied then it becomes less useful as a
denominator of other goods. \\

\subsection{Stablecoins}

\noindent A stablecoin is a cryptocurrency designed for price stability, such that it can function as
both a medium of exchange and unit of account. It should ideally be as effective for making payments
as fiat currencies like the US Dollar, but still retain the desirable characteristics of Bitcoin;
transaction immutability, censorship resistance and decentralisation. \\

\noindent Cryptocurrencies are in these ways a far better form of money, but have been significantly hindered
in their adoption by the fact that as decentralised systems, they have had relatively inflexible internal
monetary policies. Hence stability continues to be one of the most valuable and yet the most elusive
characteristics.
The fact that we have yet to achieve stability in cryptocurrencies without resorting to extreme
centralisation should by no means be taken as evidence that this problem is insurmountable.
The reality is that the technology to create alternative monetary policies within
cryptoeconomic systems has only existed for a few years. Clearly, significant research into stable monetary
frameworks for cryptocurrencies is required. \\

\subsection{Achieving Stability}

\noindent A viable, autonomous, and decentralised stablecoin is a \textit{sine qua non} for achieving a
decentralised economy.
Such a token is a challenge to design. Central banks have near complete discretion over their money supplies;
a great deal of effort and wealth are expended in supporting the stability of state-backed fiat currencies.
And yet, there are many examples where these efforts have utterly failed.
Stability is far harder in cryptoeconomic systems because their behaviour must be almost completely
defined at the start, which means that there can be very little consideration for events that are unpredicted.
We must likely resign ourselves to the fact that these systems will be only functional under certain conditions.
However, if those conditions are well understood, and the cost of destabilising the system is greater than
any derived benefit, we might ensure that the utility of the system outweighs the risks of participating in it. \\

\noindent With Havven we have accepted that such an approach to building an \textit{intrinsically} stable
currency is likely out of reach in the short term.
If economists cannot agree on the optimal approach to monetary policy, it seems unlikely that a system
constructed with the current orthodoxies baked in could succeed in the long term.
Instead we prefer simply to map the value of a stable token to that of the global reserve currency, the USD.
In this way we provide \textit{nominal} stability, by relying on the stability of the underlying
fiat currency.
This peg, stable as measured with respect to a single external asset, is more achievable, though it has
an obvious limitation: it is only as good as that external asset.
For example, ff the US dollar is subject to hyperinflation, our stable token will be similarly impacted.
However, we do not consider this to be a severe problem, given the historically greater stability of
fiat currencies, and that any other asset can be trivially substituted for USD at any time. \\

\subsection{Havven}
\noindent The mechanism Havven uses for maintaining its peg relies on two linked tokens and a
complex of incentives for stability:

\paragraph{Nomin} The stablecoin itself, whose supply floats. Its price measured in fiat currency should be relatively stable.
Other than price stability, the system should also encourage some adequate level of liquidity for nomins
to act as a useful medium of exchange.

\paragraph{Curit} The collateral token, whose supply is static.
The capitalisation of the curits in the market reflects the system's aggregate value, and the reserve
which backs the stablecoin. Thus, users who hold curits take on the role of maintaining the peg. \\

\noindent Each holder of curits is granted the right to issue a value of nomins in proportion to the USD value
of the curits they hold and are willing to place into escrow. If the user wishes to redeem their escrowed curits, they must
present the system with nomins in order to free their curits and trade them again.
The holders of this token provide both collateral and liquidity, and in so doing assume some
level of risk. To compensate this risk, nomin-issuers will be rewarded with fees the system levies
automatically as part of its normal operation.

In this manner, the system incentivises the issuance and destruction of nomins so that the value of
the nomin pool expands and contracts in proportion with the total value of curits backing them.
If the curit price changes, then the volume of the token pool changes with it.
On the other hand, if the nomin price changes exogenously, then the system is designed to provide
incentives for actors to counteract that change. \\

\subsection{What Havven is designed to solve}

\noindent Havven has been designed with two criteria in mind; not compromising decentralisation for stability,
and to be stable with respect to only an external proven medium of exchange.
The first tenet should be fairly self-evident to anyone who believes in the decentralised nature of blockchain.
The second is predicated on the fact that so long as the world is dominated by fiat, a solution that
demonstrates incremental utility over both fiat and existing cryptocurrencies as a medium of exchange will
significantly improve cryptocurrency adoption. For the end user, whether a stablecoin is intrinsically stable
or simply mapped to an external value is somewhat irrelevant.
Thus we chose a path that represents an incremental technical improvement, but that when achieved will
unlock a myriad of use cases currently not possible due to the limitations of the Bitcoin monetary policy. \\

\subsection{Havven, a stablecoin system}

\noindent We intend to design a decentralised stablecoin; a cryptocurrency which can act as a medium of
exchange by maintaining a sufficiently stable price relative to some external denomination,
but that does not require a trusted third party to maintain the stability.\\

\noindent  In his discussion of Hayek money~\cite{ametrano2016hayek}, Ametrano correctly makes the point that
Bitcoin serves the purpose of crypto-gold much better than it does crypto-unit-of-account due to its volatility
and constrained supply. By contrast, governments -- which mint their own currencies -- can and do execute
discretionary stabilisation policies to manipulate the circulating supply. This kind of powerful lever is not
available to Bitcoin and other supply-constrained currencies of its type, but a similar system whose monetary
policy is algorithmically countercyclical rather than deflationary could inherit the desirable characteristics
of both monetary paradigms. It should be possible, by automatic means, to incentivise the issuance and
destruction of tokens according to demand. In this way, users of such a currency would be allowed to
capitalise it while the system automatically seeks to expand and contract the money supply as its backing
reserve fluctuates in value. By this mechanism we might produce a more perfect currency where supply floats
with necessity, but which is not prone to debasement and other issues commonly associated with either
inflationary or deflationary forms of money. Ideally, we also seek to remove some of the distortions created
by traditional monetary policy, which, when it is expansionary, shrinks the purchasing power in every account
which is not a direct beneficiary of that policy.\\

\noindent While we believe that ultimately such a system of intrinsic stability can be constructed, we see such
an achievement as relatively far away from being practicable. As discussed above, we also believe that to the
end user the distinction is somewhat irrelevant, we therefore take an approach which we believe will be simpler
to implement and model; and that will result in practically the same outcome. Hence, we propose a system of two
related tokens which serve distinct purposes as a part of the Havven stablecoin system.

\paragraph{Curit} The collateral token. Users who hold curits take on the role of maintaining the peg. 
The holders of this token are providing collateral for the stablecoin, and in so doing, assuming some level of
risk. To compensate this risk, curit-holders will be rewarded with fees the system levies automatically as part
of its normal operation. The capitalisation of the curits in the market reflects the system's aggregate value.

\paragraph{Nomin} The stablecoin. The system intends the price of each nomin to be stabilised in terms of some
external, relatively-stable currency or commodity basket. Each holder of curits is granted the right to issue
nomins, in proportion to the value of the curits they hold and are willing to place into escrow.
If the user wishes to redeem their escrowed curits, they must present the system with nomins in order to free
their curits and trade them again. Other than price stability, the system should also encourage some adequate
level of liquidity for nomins to act as a useful medium of exchange. \\

\noindent Clearly, the introduction of a new cryptocurrency, in isolation, offers no additional value given
the existing and established alternatives such as Bitcoin and Ether. Havven thus seeks to derive value from the
addition of \textbf{stability} to its inherited properties as a modern cryptocurrency. \\

\noindent The Havven stablecoin system is akin to representative money in the sense that the fungible ``nomin''
tokens represent some value held in reserve. We define the curit to be the token of backing value as this is
both the start and end point of using the Havven mechanism; curits develop intrinsic value given their ability
to maintain stability (through nomins) with an external denomination. Hence, nomins have no intrinsic value
because we define curits as carrying the value associated with being able to provide a functioning stable
medium of exchange. \\

\noindent Havven however is not representative money as we have traditionally known it. Historical instantiations,
such as the gold standard which allowed anyone to claim against the reserve, caused exacerbations in times of
economic turmoil. Given that it does not need to act as the primary currency in the market, Havven is relieved
of any pressure to respond and correct for macroeconomic market issues. We leave such manipulations of the money
supply to the whims of central banks, for good or ill. Thus Havven is at its simplest a bridge between fiat and
crypto, a hybrid of the two technologies and thus for numerous use cases superior to both. But it bears
explicitly stating that whatever monetary policy is applied to the external value that nomins are mapped to will
flow through to the system, such that if the USD is significantly devalued through inflation, so too will the
nomin. In this scenario, the value of curits against the USD will increase and more nomins will be able to be
issued against that value, so long as they are denominated in USD. \\

\noindent The Havven system is designed such that the issued stablecoin (nomins) is both denominated in and
mapped to an external store of value. Throughout this paper we use USD as the reference, however this could
be any external and appropriately fungible asset, such as a commodity or fiat currency, (note, denominations
in other cryptocurrencies are not necessary as these already benefit from the features Havven is implementing
for the external denominator). \\

\subsection{Use cases for Stablecoins}

\noindent Nomins in the early stages of the system are anticipated to be used as a means to hedge against
cryptocurrency volatility and as a settlement layer. For example, centralised cryptocurrency exchanges may
use the Havven system to settle between themselves, expressing the value of the settled funds/assets in USD
(for nomins denominated in USD). Later, with greater maturity and scalability of Ethereum, Havven may be
extended for more general purpose use. \\


\noindent More information can be found on havven.io or by reading the position paper.

\pagebreak
