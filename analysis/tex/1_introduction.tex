\section{Introduction}

Introduction. \\

\subsection{What is money?}

There are three primary functions of money; to act as a unit of account, a medium of exchange and as a store of value. In addition to the three functions money should ideally exhibit these characteristics; durability, portability, divisibility, uniformity, limited supply, and acceptability. Over the last few decades money has become almost invisble as payment technology has advanced, because of this it is often lost upon users of money that it is itself a technology. Money, like all technologies can be improved upon, in this case that means improving the perfomance of the three functions, or improving the six characteristics.

\subsection{Bitcoin as money}

Bitcoin as a technological improvement on existing forms of money is impressive because it manages to simultaneously improve durability, portability, and divisibility. Further, it does so without requiring the support of a nation state from which to derive its value. The bitcoin supply is, therefore,  not subject to control by a centralised entity. This fixed monetary policy means that increased adoption has tended to drive the price up over time allowing bitcoin outperform other forms of money as a store of value, precisely because it is not subject to debasement and devaluation. Unfortunately this fixed monetary supply creates the potential for volatility in the short term because there is no mechanism within the Bitcoin system that can monitor or adjust to changing demand for the currency. Thus it has tended to be a poor medium of exchange and and even worse unit of account. In order for something to perform well as a medium of exchange or unit of account it must remain relatively stable against other goods and services, because money is ultimately a good that other goods are denominated in. If the price of money as a good is too varied then it becomes less useful as a denominator of other goods. \\

\subsection{What is a stablecoin}

\noindent A stablecoin is a cryptocurrency designed for price stability, such that it can function as both a medium of exchange and unit of account. It should ideally be as effective for making payments as fiat currencies like the US Dollar, but still retain the characteristics of bitcoin; transaction immutability, censorship resistance and decentralisation. \\

\subsection{Why stablecoins}

\noindent Cryptocurrencies, while in some ways a far better form of money, have been significantly hindered in their adoption by the fact that as decentralised systems they have had relatively inelastic internal monetary policies, thus stability continues to be one of the most valuable and yet the most elusive characteristics. The fact that we have yet to achieve stability in cryptocurrencies without resorting to extreme centralisation should by no means be taken as evidence that this problem is insurmountable. The reality is that the technology to create alternative monetary policies within cryptoeconomic systems has only existed for a few years. There are numerous structural and cultural reasons why stability has been somewhat ignored in favour of other challenging problems, these will be outlined later in the paper, but suffice to say we believe that significant research into stable monetary frameworks for cryptocurrencies is required. \\

\noindent Outline the current issues around why stablecoin (particularly the centralisation risks that Kain called out) ***I think we move this to a leter section.*** \\

\subsection{Intrinsic versus Nominal Stability}

\noindent Stability is a complex goal, and there are many different approaches to acheiving it, but there is no question that a viable, autonomous and decentralised system that supports the issuance of an intrinsically stable token is a key element in the progression towards a dencentralised economy. Unfortunately while there is no evidence demonstrating this approach is not possible, such a system is still extremely challenging to design. Central banks have almost complete discretion over their money supplies and vast numbers of man hours are expended in supporting the stability of state based fiat currencies and yet there are many examples where these efforts have failed utterly. Stability is far harder in cryptoeconomioc systems, because a cryptoeconomic system must be almost completely defined at the start, which means that there can be very little consideration for events that are unpredicted. We must likely resign ourselves to the fact that such a sytem will be only functional within certain narrow parameters. This is not in itself neccesarily an issue, provided the paramters within which the system fucntions are well understood and the cost of destablising the system is sufficient such that the utility of the system outweighs the risks of participating in it. \\

\noindent With Havven we have accepted that such an approach is likely to be out of reach in the short term. If economists cannot agree on the optimal approach to monetary policy, it seems unlikely that a system contructed with the current state of knowledge could succeed in the long term. We have therefore, decided to take an alternative approach, to simply provide nominal stability by mapping the value of a stable token to an external currency, initially the USD. This significantly reduces the surface area of the system, since we do not need our stable token to be stable in and of itself, we only need it to be stable with respect to an external measure. So if the USD becomes subject to hyperinflation our stable token will be similarly impacted. The mechanism by which we achieve this is described below, but in short, we find the value for the entire system in USD and use the value of the system to issue a token denominated in USD backed by the collateral of system itself. We thus remove the need for the system to monitor demand for the currency, it must simply be able to determine the price of the entire system and ensure that the total value of circulating currency remains below some pre-defined threshold.\\

\subsection{What Havven is designed to solve}

\noindent Havven has been designed with two criteria in mind; not compromising decentralisation for stability, and to be stable only with respect to an external proven medium of exchange, not to itself be intrinsically stable. The first tenent should be fairly self evident to anyone who believes in the decentralised nature of blockchain. The second is predicated on the fact that so long as the world is dominated by fiat, a solution that demonstrates incremental utility over both fiat and existing cryptocurrencies as a medium of exchange will significantly improve crypto adoption. For the end user, whether a stablecoin is intrinsically stable or simply mapped to an external value is somewhat irrelevant. Thus we chose a path that represents an incremental technical improvement, but that when achieved will unlock a myriad of use cases currently not possible due to the limitations of the Bitcoin monetary policy. \\

\subsection{Havven, a stablecoin}

\noindent We intend to design a decentralised stablecoin; a cryptocurrency which can act as a medium of exchange due to it possessing a sufficiently stable price relative to other goods and services, but that does not require a trusted third party to maintain the stability of the system.\\

\noindent  In his discussion of Hayek money, Ametrano correctly makes the point that Bitcoin serves the purpose of crypto-gold much better than it does crypto-unit-of-account, due to its volatility and constrained supply.~\cite{ametrano2016hayek} By contrast, governments, who mint their own currencies, can and do execute discretionary stabilisation policies to manipulate the circulating supply. This kind of powerful lever is not available to Bitcoin and other supply-constrained currencies of its type, but a similar system whose monetary policy is algorithmically countercyclical rather than deflationary could inherit the desirable characteristics of both. It should be possible however, by automatic means, to incentivise the issuance and destruction of tokens according to demand. In this way, users of such a currency would be allowed to capitalise it while the system automatically seeks to expand and contract the money supply as its backing reserve fluctuates in value. By this mechanism we might produce a more perfect currency where supply floats with necessity, but which is not prone to debasement by selfish or misguided actors, or other issues commonly associated with inflationary forms of money. We should also seek to ideally remove some of the distortions created by traditional monetary policy, which, when it is expansionary, shrinks the purchasing power in every account which is not a direct beneficiary of that policy.\\

\noindent Whilst we believe that ultimately such a system of intrinsic stability can be constructed, we see such an achievement as relatively far away from being implementable. As discussed above, we also believe that to the end user the distinction is somewhat irrelevant, we therefore take an approach which we believe will be far simpler to implement and model and that will result in practically the same outcome. Hence, we propose a system of two tokens which serve distinct purposes as a part of the Havven stablecoin system:

\paragraph{Curit} The collateral token. Users who hold Curits take on the role of maintaining the system. The holders of this token are providing collateral for the stablecoin, and in so doing, assuming some level of risk. To compensate this risk, Curit-holders will be rewarded with fees the system levies automatically as part of its normal operation. The capitalisation of the Curits in the market reflects the system's aggregate value.

\paragraph{Nomin} The stablecoin. The system intends the price of each Nomin to be stabilised in terms of some external, relatively-stable currency or commodity basket. Each holder of Curits is granted the right to issue Nomins, in proportion to the value of the Curits they hold and are willing to place into escrow. If the user wishes to redeem their escrowed Curits, they must present the system with Nomins, in order to freely trade them again. Other than price stability, the system should also encourage some adequate level of liquidity for Nomins to act as a useful medium of exchange. \\

\noindent Clearly, the introduction of a new cryptocurrency, in isolation, offers no additional value given the existing and established alternatives, such as Bitcoin and Ether. Havven then seeks to derive value from the addition of \textbf{stability} to its inherited properties as a modern cryptocurrency. \\

\noindent The Havven stablecoin system is akin to representative money in the sense that the fungible "Nomin" tokens of the Havven system represent some value held in reserve. We define the Curit to be the token of backing value as this is both the start and end point of using the Havven mechanism; Curits develop intrinsic value given their ability to maintain stability (through Nomins) with an external denomination. Hence, Nomins have no intrinsic value as we define Curits as carrying the functional value associated with being able to provide a stable medium of exchange. \\

\noindent However, Havven is not representative money as we have traditionally known it. Historical instantiations, such as the gold standard which allowed anyone to claim against the reserve, caused exacerbations in times of economic turmoil. Havven given that it does not need to act as the primary currency in the market is relieved of any pressure to respond and correct for macroeconomic market issues. We leave such manipulations of the money supply to the whims of central banks, for good or ill. Thus Havven is at its simplest a bridge between fiat and crypto, a hybrid of the two technologies and thus for numerous use cases superior to both. But it bears explicitly stating that whatever monetary policy is applied to the external value that Nomins are mapped to, will flow through to the system, such that if the USD is significantly devalued through inflation the value of curits against the USD will increase and more Nomins will be able to be issued against that value so long as they are denominated in USD. \\

\noindent The Havven system is designed such that the stable currency issued (Nomins) is both denominated in and mapped to an external store of value. Throughout this paper we use USD as a the reference, however this could be any external and appropriately fungible asset, such as a commodity or fiat currency, (note, denominations in other cryptocurrencies are not necessary as these already benefit from the features Havven is implementing for the external denominator). \\

\noindent Nomins in the early stages of the system are anticipated to be used as a means to hedge against cryptocurrency volatility and as a settlement layer. For example, centralised cryptocurrency exchanges may use the Havven system to settle between themselves, expressing the value of the settled funds/assets in USD (for Nomins denominated in USD). Later, with greater maturity and scalability of Ethereum, Havven may be extended for more general purpose use. \\

\subsection{Use cases for Stablecoins}
\noindent USE CASES - Kain to add. \\

\noindent More information can be found on havven.io or by reading the position paper.

\pagebreak
