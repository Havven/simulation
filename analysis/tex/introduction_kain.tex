\section{Introduction}

\subsection{Money and Cryptocurrencies}

There are three primary functions of money; to act as a unit of account, a medium of exchange
and as a store of value. In addition, money should ideally exhibit durability,
portability, divisibility, uniformity, limited supply, and acceptability.
Money has become almost invisible over the past few decades as payment technology has advanced.
Because of this, it is often lost upon users of money that it is itself a technology that can be
improved. Specifically, this means improving the performance of our six desirable properties. \\

\noindent Bitcoin as a technological improvement on existing forms of money is impressive because it manages to simultaneously improve durability, portability, and divisibility.
Further, it does so without requiring the enforcement of a nation state from which to derive its value.
The Bitcoin supply is, therefore, not subject to control by any central authority. \\

\noindent This fixed monetary policy means that increased adoption has tended to drive the price up over time,
allowing Bitcoin to outperform other forms of money as a store of value, precisely because it is not
subject to debasement and devaluation. Unfortunately this fixed monetary supply creates the potential
for volatility in the short term because there is no mechanism within Bitcoin that can monitor or
adjust to changing demand for the currency. \\

\noindent Thus it has tended to be a poor medium of exchange and an even worse unit of account.
In order for something to perform well as a medium of exchange or unit of account it must remain
relatively stable against other goods and services because money is ultimately a good that other goods
are denominated in. If the price of money as a good is too variable then it becomes less useful as a
denominator of other goods.

\subsection{Stablecoins}

\noindent A stablecoin is a cryptocurrency designed for price stability, such that it can function as
both a medium of exchange and unit of account. It should ideally be as effective for making payments
as fiat currencies like the US Dollar, but still retain the desirable characteristics of Bitcoin;
transaction immutability, censorship resistance and decentralisation. \\

\noindent Cryptocurrencies are in these ways a far better form of money, but have been significantly hindered
in their adoption by volatility caused by the fact that as decentralised systems, they have tended to have relatively inflexible internal
monetary policies. Hence stability continues to be one of the most valuable and yet the most elusive
characteristics for the technology. Clearly, the ability to create alternative monetary policies within
cryptoeconomic systems is still new, and significant research into stable monetary
frameworks for cryptocurrencies is required.

\subsection{Havven}
\noindent The Havven system is a novel form of representative money where there is no requirement for a physical asset, thus the problem of trust and custodianship is removed. The asset we use to back our stablecoin is the system itself. This is achieved because the Havven system generates fees from users who transact in the stablecoin; participants who hold the collateral token receive these fees and thus the system rewards those who actively participate in maintaining the system and charges those who utilise the system. Because we have created a system that generates cash flow for participants we now have an asset which has a defined market value and can be used as the collateral to support the stablecoin. The key to this is that the value of the system is measured in USD. This allows us to issue a stablecoin which can be presented and redeemed for a percentage of the collateral tokens valued at 1 USD. Backing a stablecoin in this way is beneficial because such a cryptoeconomic system does not require trust in a centralised party; each participant has full transparency over how many tokens have been issued against the available collateral at all times. \\

\noindent The two linked tokens and the complex of incentives for stability are defined below:

\paragraph{Nomin} The stablecoin itself, whose supply floats. Its price measured in fiat currency should be relatively stable.
Other than price stability, the system should also encourage some adequate level of liquidity for nomins
to act as a useful medium of exchange. The Nomin has value because it can be redeemed directly from the system for a fraction of curits worth 1 USD.

\paragraph{Curit} The collateral token, whose supply is static.
The capitalisation of the curits in the market reflects the system's aggregate value, and the reserve
which backs the stablecoin. Thus, users who hold curits take on the role of maintaining stability. \\

\noindent Each holder of curits is granted the right to issue a value of nomins in proportion to the USD value
of the curits they hold and are willing to place into escrow. If the user wishes to redeem their escrowed curits, they must
present the system with nomins in order to free their curits and trade them again.
The holders of this token provide both collateral and liquidity, and in so doing assume some
level of risk. To compensate this risk, such nomin-issuers will be rewarded with fees the system levies
automatically as part of its normal operation. \\

\noindent In this manner, the system incentivises the issuance and destruction of nomins so that the value of
the nomin pool expands and contracts in proportion with the total value of curits backing them.
If the curit price changes, then the volume of the token pool changes with it.
On the other hand, if the nomin price changes exogenously, then the system is designed to provide
incentives for actors to counteract that change. \\

\noindent The Havven stablecoin system is a form of representative money in the sense that the fungible nomin
tokens represent some value held in reserve. We define the curit to be the token of backing value given their ability to maintain stability (through nomins) with an external denomination. Hence, nomins only have value insomuch as they can be redeemed for Curits and carry the value associated with being able to provide a functioning stable medium of exchange. \\

\noindent Havven is not subject to one of the historical limitations of representative money in that we are able to significantly overcapitalise the circulating currency without any macroeconomic consequences. Given that it does not need to act as the primary currency in the market, Havven is relieved of any pressure to respond and correct for macroeconomic market issues. Thus Havven acts as a bridge between fiat currency and
cryptocurrency - a hybrid of the two technologies which inherits the monetary policies applied to the external denomination.

\noindent The Havven system is designed such that the nomin is both denominated in and
mapped to an external store of value. Throughout this paper we use USD as the reference, however this could
be any external and appropriately fungible asset, such as a commodity or fiat currency. Note that denominations
in other cryptocurrencies are not necessary as these already benefit from the features Havven is implementing
for the external denominator.

\subsection{Rationale}

\noindent  In his discussion of Hayek money~\cite{ametrano2016hayek}, Ametrano correctly makes the point that
Bitcoin serves the purpose of crypto-gold much better than it does crypto-unit-of-account due to its volatility
and constrained supply. By contrast, governments -- which mint their own currencies -- can and do execute
discretionary stabilisation policies to manipulate the circulating supply. This kind of powerful lever is not
available to Bitcoin and other supply-constrained currencies of its type, but a similar system whose monetary
policy is algorithmically countercyclical rather than deflationary could inherit the desirable characteristics
of both monetary paradigms. Hence it should be possible, by automatic means, to incentivise the issuance and
destruction of tokens according to demand. In this way, users of such a currency would be allowed to
capitalise it while the system automatically seeks to expand and contract the money supply as its backing
reserve fluctuates in value. \\

\noindent There are many applications which Bitcoin's inherently deflationary monetary policy and
volatility presently make impossible. any token which is able to demonstrate an increment
in utility on these fronts over both fiat and cryptocurrencies will significantly
enhance the uptake of cryptoeconomic technology. Clearly, the introduction of a new cryptocurrency in isolation offers no additional value given
the existing and established alternatives such as Bitcoin and Ethereum. Havven thus seeks to derive value
from the addition of \textbf{stability} to its inherited properties as a modern cryptocurrency.
It is designed to provide a practical medium of exchange, without compromising the benefits that decentralisation offers in order to substantially improve the technology of money.

\pagebreak
