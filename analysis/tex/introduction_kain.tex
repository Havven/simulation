\section{Introduction}

\subsection{Money and Cryptocurrencies}

There are three primary functions of money; to act as a unit of account, a medium of exchange
and as a store of value. In addition, money should ideally exhibit durability,
portability, divisibility, uniformity, limited supply, and acceptability.
Money has become almost invisible over the past few decades as payment technology has advanced.
Because of this, it is often lost upon users of money that it is itself a technology that can be
improved. Specifically, this means improving the performance of our six desirable properties. \\

\noindent Bitcoin is an impressive technological advancement upon existing forms of money which
simultaneously improves durability, portability, and divisibility.
Further, it does so without requiring the enforcement of a nation state from which to derive its value.
The Bitcoin supply is, therefore, not subject to control by any central authority. \\

\noindent It is precisely its fixed monetary policy which has protected Bitcoin from debasement or
devaluation, allowing it to outperform other forms of money as a store of value. Increased adoption
has tended to drive the price up over time; unfortunately the fixed money supply has created the
potential for short-run volatility, because there is no mechanism within Bitcoin that can
adjust to changing demand for the currency. \\

\noindent Thus it has tended to be a poor medium of exchange and an even worse unit of account.
In order for something to perform well as a medium of exchange or unit of account it must remain
relatively stable against other goods and services because money is ultimately a good that other goods
are denominated in. If the price of money as a good is too variable then it becomes less useful as a
denominator of other goods.

\subsection{Stablecoins}

\noindent A stablecoin is a cryptocurrency designed for price stability, such that it can function as
both a medium of exchange and unit of account. It should ideally be as effective for making payments
as fiat currencies like the US Dollar, but still retain the desirable characteristics of Bitcoin;
transaction immutability, censorship resistance and decentralisation. \\

\noindent Cryptocurrencies are in these ways a far better form of money, but have been significantly hindered
in their adoption by volatility caused by the fact that as decentralised systems, they have tended to have relatively inflexible internal
monetary policies. Hence stability continues to be one of the most valuable and yet the most elusive
characteristics for the technology. Clearly, the ability to create alternative monetary policies within
cryptoeconomic systems is still new, and significant research into stable monetary
frameworks for cryptocurrencies is required.

\subsection{Havven}

\noindent The Havven system is a novel form of representative money where there is no requirement for a physical
asset, thus the problem of trust and custodianship is removed. The asset we use to back our stablecoin is 
the pool of reserve tokens, contained within the system itself. These tokens reflect participation in the system,
and are a proxy for its value. Havven generates fees from users who transact in the stablecoin;
and distributes them among the holders of the reserve token, compensating them for underpinning the system.
Thus Havven rewards those who actively participate in maintaining it and charges those who passively utilise it.
To maintain stability, Havven includes market incentives that manage the supply of the exchange token such that
its price mirrors that of the asset it tracks.

\noindent Because we have created a system that generates cash flow for participants we now have an asset which has a defined market
value and can be used as the collateral to support the stablecoin. The key to this is that the value of the
system is measured in USD. This allows us to issue a stablecoin which can be presented and redeemed for a
percentage of the collateral tokens valued at 1 USD. Backing a stablecoin in this way is beneficial because
such a cryptoeconomic system does not require trust in a centralised party; each participant has full
transparency over how many tokens have been issued against the available collateral at all times. \\

\noindent The two linked tokens and the complex of incentives for stability are defined below:

\paragraph{Nomin:} The stablecoin itself, whose supply floats. Its price measured in fiat currency should be relatively stable.
Other than price stability, the system should also encourage some adequate level of liquidity for nomins
to act as a useful medium of exchange. The nomin has value because it can be redeemed directly from the system for a fraction of curits worth 1 USD.

\paragraph{Curit:} The collateral token, whose supply is static.
The capitalisation of the curits in the market reflects the system's aggregate value, and the reserve
which backs the stablecoin. Thus, users who hold curits take on the role of maintaining stability. \\

\noindent Each holder of curits is granted the right to issue a value of nomins in proportion to the USD value
of the curits they hold and are willing to place into escrow. If the user wishes to redeem their escrowed curits, they must
present the system with nomins in order to free their curits and trade them again.
The holders of this token provide both collateral and liquidity, and in so doing assume some
level of risk. To compensate this risk, such nomin-issuers will be rewarded with fees the system levies
automatically as part of its normal operation. \\

\noindent This issuance mechanism allows nomins to act as a form of representative money, where 
each nomin represents a share in the curit value held in reserve. Nomins derive value insofar as they provide
a superior medium of exchange, and are effectively redeemable for a constant value
of the denominating asset. In this paper, we use USD as this asset, but this could be any external
and appropriately fungible asset, such as a commodity or a fiat currency.  \\

\noindent In this manner, the system incentivises the issuance and destruction of nomins so that the value of
the nomin pool expands and contracts in proportion with the total value of curits backing them.
If prices changes exogenously, then the system is designed to provide incentives for actors to
counteract that change. \\

\noindent The Havven system is relieved of the obligation to respond to major macroeconomic conditions, 
as it benefits from the stabilisation efforts of large institutions acting in fiat markets.
As Havven has the freedom to significantly overcollateralise its pool of circulating currency, it
insulates itself against dramatic corrections in the curit market.
Thus Havven acts as a bridge between fiat currency and cryptocurrency - a hybrid of the two technologies which possesses
the advantages of both. \\

\subsection{Rationale}

\noindent  In his discussion of Hayek money~\cite{ametrano2016hayek}, Ametrano correctly makes the point that,
due to its volatility and constrained supply, Bitcoin serves the purpose of crypto-gold much better than it
does crypto-unit-of-account. By contrast, governments -- which mint their own fiat currencies -- can and do execute
discretionary stabilisation policies to manipulate the circulating supply. This powerful lever is not
available to Bitcoin and other supply-constrained currencies of its type, but a cryptocurrency whose monetary
policy is algorithmically countercyclical rather than deflationary could inherit the desirable characteristics
of both monetary paradigms. It should be possible to automatically provide incentives the issuance and
destruction of tokens according to demand. Users of such a currency would be allowed to back it while
the system automatically seeks to expand and contract the money supply as its backing
reserve fluctuates in value. \\

\noindent Clearly, the introduction of a new cryptocurrency in isolation offers no additional value given
the existing and established alternatives such as Bitcoin and Ethereum. Havven thus seeks to derive value
from the addition of \textbf{stability} to its inherited properties as a modern cryptocurrency.
It is designed to provide a practical medium of exchange, without compromising the benefits that
decentralisation offers in order to substantially improve the technology of money.
There are many applications which Bitcoin's inherently deflationary monetary policy and
volatility presently make impossible: any token which is able to demonstrate an increment
in utility on these fronts over both fiat and cryptocurrencies will significantly
enhance the uptake of cryptoeconomic technology.

\pagebreak
