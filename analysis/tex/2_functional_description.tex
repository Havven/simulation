\section{Functional description}

Havven works by providing a set of market incentives that support the mapping of Nomin value to an external asset.

\subsection{Stability design considerations}

Fundamentally, we wish to configure the system such that it incentivises the desired properties of a stablecoin. These are:
\begin{itemize}
    \item 1. Value stablisiation
    \item 2. Value transfer
\end{itemize}

This paper focuses on value stabilisation as the key enabler for a better form of money; once we have this, we assume that we get value-transfer (market share for the currency) for free.


Let us consider the various ways in which one can maintain a stable value relative to a fiat currency? The question we wish to answer is, "How can we \textbf{control the value} of the cryptocurrency such that the price of one unit of the stablecoin matches the price of one unit of the denominating currency?" This is a challenging scenario because there are multiple related forces at work on the price of each currency. We consider these as two independent groups: "market forces" and "control forces". \\

\noindent \textbf{Market} forces represent supply and demand. These are necessarily different for both currencies, otherwise the value of both currencies would move in unison. \\

\noindent \textbf{Control} forces then are the controls one is able to apply over a currency to affect its value, such as an inflation rate or a buy-back scheme. \\

\noindent Our price mapping then should seek to tune the control forces such that one unit of a control currency equals one unit of the denominating currency. We assume that the forces for one currency are independent of the forces for another. \\

\noindent So what are the mechanisms we can apply to control the value of a currency? We consider:

\begin{itemize}
	\item Issuing new currency to increase supply (inflation)
	\item Buying back existing currency to decrease supply (deflation)
	\item Unilateral balance control (changing account balances to maintain a stable buying power)
	\item \textit{Others?}
\end{itemize}

\noindent Unilateral balance control, such as that as described in Amentrano's paper on Hayek Money, is discounted on the basis that individual's balances being directly modified would be unpalatable to the general population. \\

This leaves us with simply the forces relating to modifying supply...

We review the key incentive drivers in the design of an economically stable cryptocurrency. \\

\begin{itemize}
	\item Fees
	\item Supply control
	\item Capital growth
	\item Bonds
\end{itemize}

\noindent We consider the following significant questions:

\begin{itemize}
    \item How do we incentivise actors to contribute to system liquidity?
    \item How should Nomins be created and destroyed?
    \item Should fees only be given to those who have actually issued Nomins?
    \item Are transfer fees charged in only Nomins? Would actors not then just try to convert to Curits and exchange there?
    \item How do we select a utilisation ratio? What is its curve?
    \item \textit{@anton What is this one?} Utilisation ratio needs to have distinct parts. Number of escrowed Curits is not quite the same as the number of issued nomins.
    \item Should we allow the system to maintain a pool of Curits/Nomins that it itself can buy/sell?
    \item What if the system didn't burn the Nomins that were handed back when Curits were redeemed?
    \item Should Nomins, or Curits, be serialised?
    \item Should newly created Nomins go to escrowed Curit accounts, or be sold?
\end{itemize}

\subsection{Description of mechanism}

Curit holders have an ongoing option to commit to an escrow of their collateral token in return for an amount of Nomins valued at some fraction of the value of the Curits, denominated in USD. We call this fraction the Utilisation Ratio, (or later more formally, the Issuance Ratio), $ U $. \\

\noindent The Curit holder can then sell their Nomins for ETH for any price (or we can force a sale at the correct ETH price), and retain the ETH.


\subsection{Alternative approaches}

\noindent An autonomous system which mimics the actions of a central-bank, wherein reserves of cryptocurrency are held for the purposes of currency buy-back was also considered. This approach involves a complex system of managing bonds and Nomin issuance as coupons to bond-holders (similar to Basecoin), whilst also acting as a buyer of last resort for all Nomins sold under the peg beyond some threshold. This approach was discounted due to the need to anticipate complex market counter-algorithms to support the peg. \\

\noindent The Havven mechanism is a novel and simpler approach which uses only open market incentives for economically rational participants to bring stability to the exchange token. \\

\textit{(SGB: escrowing Curits and selling them for nomins by a Curit holder creates demand for when the holder needs to buy-back the nomins to get his Curits back. If we don't do this and hold the ether in reserve, we can automatically buy-back nomins much more efficiently. This is probably far more stable than having individuals control and possible holding onto the nomin supply, thereby creating large spikes in both demand and supply; a spike in demand could be felt when there's a run on Nomins to release Curits, this could be compensated for if other efficient market actors then escrowed their Curits to sell nomins at the higher price, but in the case that there are no more nomins to issue (all Curits are escrowed), the price of Nomins could skyrocket.)}

\subsection{Price discovery}

One of the key challenges with denominating a cryptocurrency in a fiat currency is the fundamental link this creates to the centralised world; when the denominating currency exists external to the blockchain ecosystem, some bridge must be built so that the system can act with knowledge of the outside world. Often, this is done by sacrificing trust; in order to reclaim system performance, we can trade some of the trustlessness of the design, such as through implementing an trusted "Oracle" service in order to gain knowledge of the external world and build a causal link. \cite{brooks2017blog}.\\

\noindent Havven achieves this causal bridge to USD through market arbitrage; on the Havven decentralised exchange (hosted on Ethereum), both Curits and Nomins are denominated in ETH. Ether, in turn, is assumed to have some instantaneous value in USD on one or more external markets. In this way, we can avoid a trusted bridge and instead reduce our price-finding requirement to the minimum viable assumption that there will always be some external market for ETH in USD (or some other chosen denomination) and that market actors will seek profit through arbitrage. \\

\noindent This method is superior to using an Oracle service pushing the exchange rate of ETH/USD at a regular cadence into the Havven decentralised exchange as it completely removes any risk associated with centralisation risks of such a process. \\

\subsection{Backing collateral}

Central to the design of the Havven money system is how the currency is backed by collateral. \\

\noindent For monetary systems that are backed by an external asset, centralisation risks are frequently encountered and are often without solution. The question arises: \\

\noindent How can one take an external asset and make it distributed such that we can mitigate the centralisation risk? \\

\noindent Finally, we arrive at the simplest and most ideal solution, being to use the system itself as the backing collateral. We can then issue an exchange token against this in a manner similar to fiat in that the exchange token has no intrinsic value, but similar to representative money as we argue that the collateralised value exists in the Curit token. \\

\noindent \textit{SGB: Critical issue: fiat works because you cannot copy a country. Havven may not work because you can easily copy the source code and start your own. So there needs to be a great deal of effort expended to gain market traction. This is where the ICO comes in and an open-ended one could work the best.} \\

\noindent The basis for Havven is an initial acceptance of the idea that the asset can be the system. The only other alternative is to use a basket of cryptocurrencies to back the exchange token, however this ultimately suffers from the same issues we are trying to prevent, namely volatility that is only mitigated through diversification of the crypto-assets chosen. \\

\noindent Finally, in backing by a real-world (external) asset, such as gold or a state currency, we encounter centralisation risks once again, including an inability to prove the existence of the backing asset (fraud risk). \\

\noindent Fundamentally, whenever you want to denominate a stablecoin in something from the real world (such as USD), then you will always need a bridge between the 'real-world' and the walled garden of the blockchain. \\

\noindent To this end, our design incorporates a special exchange (bridge) between fiat and Ether. This exchange is run as a not-for-profit by the Havven benevolent dictatorship. The price of Ether is then taken from this exchange and injected into the DEX (again, assuming scalability issues aside). The DEX then has an authoritive Ether/USD value  \\

\noindent => what happens when this exchange is shutdown by the feds? The whole system collapses. You will always need some kind of trusted source of information for the exchange rate if the denominating asset is outside the system. \\

\noindent \textit{SGB: Volume of reserves is also problematic if one is just using the backing system as collateral - this collateral pool (i.e. value of the functionality of maintaining stability plus the value of the generated fees?) is small relative to the value of Nomins on issue (i.e. what happens if the price of nomins goes up but the value of Curits stays the same?)} \\

\paragraph{Raised funds}

What happens with accumulated ICO funds beyond what is required to build and market the platform? Currently these do not "back" the platform; if the value of the fund moves, the value of Curits is insulated from this. Assuming that the value of funds raised far exceeds the funds required to build. \\

\noindent \textit{SGB: Perhaps these raised funds can be used to help stabilise the system in the same way that Basecoin does it. I.e. issue a small buyer of last resort function that perhaps takes a very small fee clip that adds to the collateral in reserve so that the system becomes more stable over time given the increasing reserves.}

\subsection{Approach}

\begin{itemize}
	\item Functionality of Curits and Nomins
	\item Initial Pricing of the value of the system:
		\subitem  1. market-found (likely to be undervalued and grow into your value)
		\subitem     2. pre-determined (\$5b)
		\subitem     3. Hybrid (capped ICO of a portion of the Curits at a value Haven specifies)
	\item 3. Ongoing pricing:
		\subitem     1. Dex - current proposal.
		\subitem     2. Rolling auction is still on the table.
		\subitem     3. Oracle
\end{itemize}

This section will provide an informal treatment of the proposed market's structure and dynamics, while section 2 will scaffold out its structure with a little more formality, including the definitions of all the important system variables. If you see an unfamiliar symbol in section 1, look up its definition in section 2.

\subsection{Investment incentives}

Why would anyone buy Curits in the first place? A potential buyer of Curits has at least three avenues for making money in Havven.

\paragraph{Capital gains due to the appreciation of Curits:}
Presumably the currency will appreciate due to a demand for Curits that is founded in the intrinsic utility of a stablecoin. Speculators will naturally be important players too.

\paragraph{Interest accrued from fees:}
If and when the price of Curits stabilises, then this may be the only long term positive-expected source of revenue. Ideally fees are set at a level where they are both high enough to be an incentive for rent-seekers to hold Curits in the long term (thus assuming the risk of providing collateral for the system) and low enough not to be a disincentive for ordinary users to transact in nomins.
It is desirable, perhaps in a future world dominated by micropayments, for these fees to be negligible for end users, while still being macroeconomically important for the system, and for those who capitalise it.

\paragraph{Arbitrage profit:}
It is the arbitrageurs who will ultimately bring the price of Nomins back into balance by a triangular circuit through Nomins, Curits, and the external (crypto or fiat) markets. They might hold Curits for a short time in order to pursue this strategy.


\pagebreak
\subsection{Fees}
There are a number of questions to be asked, and answered:
\begin{itemize}
    \item What are fees for?
    \item Who gets those fees?
    \item When can fees be levied?
    \item What macroeconomic effect does this levy have as a coin travels through the system?
\end{itemize}

\paragraph{The purpose of fees}

Fees will be redistributed to those who back the system, in order to incentivise people to capitalise it. The fee pool will be distributed periodically, for this purpose. However, if the system determines that the Nomin price is too low, then fees could be burned. If the price is too high then perhaps the system could sell these back into the system at a discounted rate.

Fee collection rate will also be a direct measure of the velocity of money in Havven. So it's in the interest of Curit holders to maximise liquidity in order to maximise their return.

\paragraph{Fee beneficiaries}

The previous assumption was that fees would simply be awarded to any holder of Curits. But then they get all the benefit with none of the risk. Although in the aggregate, it would be better for holders of Curits if everyone issued Nomins. The marginal return for any single player (who cannot issue a large fraction of all circulating Nomins) of actually issuing them would not outweigh the risk they take on in doing so. If a user can issue 1\% of circulating Nomins, then doing so will only increase their fee takings by 1\%.
Hence it it makes no sense to actually issue Nomins for any single player; so nobody will do it.

We must improve the marginal benefit of issuing Nomins into circulation in order to avoid this tragedy of the commons situation. So fees must be paid to those who issue Nomins, not just those who hold Curits.

\paragraph{Fee collection}

The system can potentially charge fees whenever any value is transferred, or any state is updated.

There are only a few circumstances that these things happen:

\begin{itemize}
    \item Nomin transfers
    \item Curit transfers
    \item Nomin issuance
    \item Curit redemption
\end{itemize}

The question is what levels to place these fees at. We might in general like to set higher Curit than Nomin transfer fees, making the stablecoin itself a lower friction market, in order to incentivise its use for exchange. Meanwhile, issuance and redemption fees will change the difficulty of entering and exiting the issuance game. \\

It's also possible for fees to float. The fee schedule could be altered dynamically in order to stabilise the system. It's even conceivable that the system could set negative fee rates if it needed to. We might charge punitive fees if a user is above the targeted utilisation ratio.

One example: if Nomin liquidity is low, meaning the system wants to incentivise issuance, then Nomin transfer fees could increase, thus having the combined effect of increasing the interest accrued by issuers, thus incentivising issuance and at the same time making it more expensive to transact in Nomins,
reducing demand and decreasing the liquidity requirements. \\

It should be pointed out that fees are antithetical to arbitrage. The higher the fee, the higher the friction, and the harder it is to make money by arbitrage. For example, if exchange fees amount to 1\% per trade, then a full arbitrage cycle between all three markets, (Nomins, Curits, and fiat) will cost in excess of 3\%. So it would not make sense to undertake arbitrage until such a time as the quoted exchange rate is misvalued by more than 3\% relative to the cross exchange rate. Hence, fees place a clear limit on the ability of arbitrage to stabilise price. Lower fees allow tighter stabilisation, within a window exactly in proportion with the fee rates themselves.


\pagebreak
\subsection{Encouraging liquidity}

\noindent It's desirable, when actors issue Nomins, that they are actually injected into the liquidity pool for their intended use, rather than be held by the same actor in order to benefit from both the receipt of fees but also the option of using those Nomins to release their Curits. The use of someone escrowing their Curits is that they provide backing for the currency flowing through the system, and so they should be rewarded for assuming this risk. In the fractional reserve system, this incentive is provided by the interest accrued upon the loans which generate money. It may be possible to adapt this system to Havven, by allowing issuers to escrow their Curits for a fixed time period, allowing the system to issue currency against that collateral, to be paid back a greater value of Nomins at a later time. \\

\noindent However, considering Havven as it stands today, there is an important question hanging over the mission of increasing the money supply. What's to stop someone issuing their Nomins, and then just holding onto them? In this manner they would accrue fees, but take on none of the risk of spending those Nomins, for they always have an instant option to liquidate their position and escape. An actor who had done the economically-desirable thing, on the other hand, who issued Nomins and then spent them, would be forced to buy Nomins in the open market in order to redeem their escrowed Curits. \\

\noindent If an issuer should not just be able to hold Nomins and accrue fees, that must also include letting them sit in another wallet they control. They should also not be able to sell their Nomins into the open market and with the proceeds buy the same value and let \textit{that} sit, only transferring it back to their main wallet once they want to flee.

\noindent But how to encourage a user to actually increase liquidity by buying goods with the Nomins they hold? 

\noindent Some level of Nomin liquidity above zero, where liquidity = flux = (average value).(average frequency). Consider also some optimum liquidity value above zero up to which diminishing returns are a factor.

The system may only ever be able to provide some level of stability within a set of thresholds (without actually backing the value of the token against the thing you're comparing it to, rocks, bottlecaps, USD.) This needs to be explicit.


\subsubsection{Non-discretionary Issuance}

One possibility is to simply provide an issuer no control over the tokens they issue. That is, when a quantity Nomins is issued, they are generated by the system, which then places a sell order at the current going rate for that quantity on an exchange on the behalf of the issuer. When the order is filled, the proceeds (in Curits, fiat, crypto, otherwise?) are remitted to the issuer.
Conversely, when a quantity of Nomins is burned, they must first be obtained from the open market. So a user would indicate an intention to burn, providing sufficient value to buy the proposed quantity of Nomins, and the system would bid for that quantity on their behalf, only liquidating the user's Curit position once they have been obtained.

\subsubsection{Motility}

But let us assume that we cannot force a user to issue and burn from the open market. We might like to encourage an issuer to spend their Nomins by other means. So we will give every account a motility score and pay fees in proportion with the product of this score and the number of tokens that account has issued.

This should be subject to common-sense obligations. The system should not be easily gamed. A user should not be able to cycle Nomins through accounts they control and collect fees. An issuer should not be able to just manipulate an account they control to have a high motility with small values and then dump a large value they want to hold into it. Ideally transferring value around repeatedly to manipulate the fee system would be expensive enough that the value lost to fees charged would outweigh the diminution of risk.

We would like to incentivise long transaction paths out of an account, and high out-degree nodes along those paths (so money is actually liquid/fungible).
We don't like short cycles. We don't like isolated subgraphs. Would be cool if the money could go into the main connected component of the transaction graph as quickly as possible, then circulate in there with high velocity.

\paragraph{Definitions}
\begin{align*}
    A \ &: \ \text{The set of all accounts} \\
    T \ &: \ \text{The multiset of all transactions; a subset of \(A \times A \times \mathbb{N}.\)} \\
    T_{a \rightarrow b} \subseteq T \ &: \ \text{The set of transactions from \(a\) to \(b\) with \(a, \ b \ \in A\).} \\
    v_t \ &: \ \text{the value of a transaction} \ t \in T. \\
    V_{a \rightarrow b} \ &:= \ \sum_{t \in T_{a \rightarrow b}}{v_t} \ \ \text{(the total value transferred from \(a\) to \(b\))} \\
    V_{a}^{in} \ &:= \ \sum_{p \in A}{V_{p \rightarrow a}} \\
    V_{a}^{out} \ &:= \ \sum_{p \in A}{V_{a \rightarrow p}} \\
    \intertext{We might interpret \((A, T)\) as a weighted multigraph of transactions, 
               with each transaction \(t \in T_{a \rightarrow b}\) corresponding to a weighted
               edge in that graph between nodes \(a\) and \(b\).
               Note that \(T_{a \rightarrow a} := \varnothing \), and hence \(V_{a \rightarrow a} = 0\)
               (accounts can't transfer to themselves).}
\end{align*}

\noindent We would like to know how likely a Nomin is to be spent soon from a given account.
The motility of the account should measure this. Considering an account \(a\), we will take
\(\mathcal{M}(a)\) to be the motility of \(a\):
\begin{align*}
    \mathcal{M}(a) &:= \sum_{p \in A}{P(a \ \text{transfers to} \ p) \cdot \mathcal{M}(p)} \\
    &= \sum_{p \in A}{\frac{V_{a \rightarrow p}}{V_{a}^{in}} \cdot \mathcal{M}(p)} \\
    &= \frac{1}{V_{a}^{in}} \sum_{p \in A}{V_{a \rightarrow p} \cdot \mathcal{M}(p)}
\end{align*}

Intuitively, if you transfer a lot of money to high-motility accounts, then your own motility is taken to be high.

\paragraph{Calculating Motility}
This will need to be calculated iteratively, and locally.
Note that \(V_{a \rightarrow p} = 0\) for \(p\) that \(a\) has never transferred to, so those accounts can be neglected.
It's probably too costly to store the value of \(V_{a \rightarrow b}\) explicitly. So we will have to eliminate this quantity in our expressions.
We will update motility scores whenever a new transaction \(t\) from \(a\) to \(b\) of value \(v_t\) is made.
\begin{align*}
    \intertext{Value into \(b\) increases, so \(\mathcal{M}(b)\) can be easily recalculated.} \\
    {V_{b}^{in}}' \ &\leftarrow \ V_{b}^{in} + v_t \\
    \mathcal{M}'(b) \ &\leftarrow \ \frac{1}{{V_{b}^{in}}'} \sum_{p \in A}{V_{b \rightarrow p} \cdot \mathcal{M}'(p)} \\
    \intertext{Meanwhile, the value transferred from \(a\) to \(b\) also increases.}
    V'_{a \rightarrow b} \ &\leftarrow \ V_{a \rightarrow b} + v_t \\
    \mathcal{M}'(a) \ &\leftarrow \ \frac{1}{V_{a}^{in}} \Big( V'_{a \rightarrow b} \cdot \mathcal{M}'(b) \ + \ \sum_{p \in A \backslash \{b\}}{V_{a \rightarrow p} \cdot \mathcal{M}'(p)}\Big)\\
    \intertext{Although these updates should also influence accounts which have (transitively) transferred into \(a\) and \(b\),
               we want to reward people for increasing liquidity today, rather than at some future time, and
               we take the motility of an account to be relatively stable after some time. As a result we will
               take \(\mathcal{M}'(p) \approx \mathcal{M}(p)\) for \(p \notin \{a, b\} \). Then:} 
    \mathcal{M}'(a) \ &\approx \ \frac{1}{V_{a}^{in}} \Big( (V_{a \rightarrow b} + v_t) \cdot \mathcal{M}'(b) \ + \ \sum_{p \in A \backslash \{b\}}{V_{a \rightarrow p} \cdot \mathcal{M}(p)}\Big) \\
    &\approx \ \frac{v_t}{V_{a}^{in}} \mathcal{M}'(b) \ + \frac{1}{V_{a}^{in}} \sum_{p \in A}{V_{a \rightarrow p} \cdot \mathcal{M}(p)} \\
    \intertext{So we will take our update step for a transaction \(t\) from \(a\) to \(b\) to be the following:}
    {V_{b}^{in}}' \ &\leftarrow \ V_{b}^{in} + v_t \\
    \mathcal{M}'(b) \ &\leftarrow \ \frac{V_{b}^{in}}{V_{b}^{in} + v_t} \mathcal{M}(b) \\
    \mathcal{M}'(a) \ &\leftarrow \ \frac{v_t}{V_{a}^{in}} \mathcal{M}'(b) \ + \mathcal{M}(a)
\end{align*}

\noindent It may also be nice to add a decay term, so that accounts that have not moved any money in a long time are taken to have a lower motility.


\pagebreak
\subsection{Utilisation ratio}

It's not clear to me exactly what purpose \(U_{max}\) serves. It certainly keeps the value of the pool of Curits below the value of the pool of Nomins, assuming there is no devaluation of a ratio more severe than \(U_{max}\) itself. However, if the system has adequate mechanisms enforce \(U\ \leq U_{max}\), then why not simply allow users to issue Nomins up to the maximum value of Curits they have escrowed? \\

\noindent A low \(U_{max}\) seems like it would place upward pressure on the price of Nomins. Consider a situation where \(U_{max} = 0.2\), and I have an impecunious friend, Jake, who owns a wallet which has issued \$20 worth of Nomins on the back of \$100 of escrowed Curits. At the moment, he has no
money, but his wallet is worth \$80, since he can burn \$20 worth of Nomins to get at those curits. So Jake should be willing to pay anywhere up to \$80 to buy enough Nomins to free up the Curits. This situation will still motivate Jake until the price of the Nomins he's issued is equal to the price of the Curits he's escrowed. That is, until the price of a Nomin is worth five times the price of a Curit.\\

\noindent Finally, let's consider the impact of the utilisation ratio on a Curit investor's value proposition. Examine the aggregate fees collected from Nomin transfers \(Ag_{nx}\), and expand out its definition:
\[Ag_{nx} = \frac{F_{nx} \cdot S_n \cdot C \cdot P_c \cdot U}{P_n}\]
This quantity is proportional with the actual utilisation ratio \(U\). The more Nomins that have been issued, the more fees are returned. So if \(U = 0.2\), then if the system would like to return a fee rate of 5\% per annum to Curit-holders, then fees to the tune of 25\% per year will have to be levied on Nomin transfers, assuming no other fees exist. This may be a little high.

\pagebreak
\subsection{Failure modes}
\subsubsection{Liquidity Trap}
\subsubsection{Trapped Currency}

\subsection{Assumptions}

\begin{enumerate}
	\item Ethereum will appropriately scale.
	\item Stability of existing cryptocurrencies will improve over time (declining utility of the stablecoin?).
	\item Unit of account will continue to be fiat for most use cases for the foreseeable future.
	\item The value of the platform is equal to the value of money raised in the ICO at that time.
   - the cost of developing the platform will be less than or equal to the amount raised.
	\item Price discovery - internal exchange, mirrors external exchanges (prices Curits in nomins, based on the value of nomins in USD).
	\item DEX follows external exchanges (arbitrage).
\end{enumerate}

\pagebreak