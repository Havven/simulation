\section{Introduction}

\subsection{Money and Cryptocurrencies}

There are three primary functions of money; to act as a unit of account, a medium of exchange
and as a store of value. In addition, money should ideally exhibit durability,
portability, divisibility, uniformity, limited supply, and acceptability.
Money has become almost invisible over the past few decades as payment technology has advanced.
Because of this, it is often lost upon users of money that it is itself a technology that can be
improved. Specifically, this means improving the performance of our six desirable properties. \\

\noindent Bitcoin is an impressive technological advancement upon existing forms of money which
simultaneously improve durability, portability, and divisibility.
Further, it does so without requiring the enforcement of a nation state from which to derive its value.
The Bitcoin supply is, therefore, not subject to control by any central authority.

This fixed monetary policy means that increased adoption has tended to drive the price up over time,
allowing Bitcoin to outperform other forms of money as a store of value, precisely because it is not
subject to debasement and devaluation. Unfortunately this fixed monetary supply creates the potential
for volatility in the short term because there is no mechanism within Bitcoin that can monitor or
adjust to changing demand for the currency.

Thus it has tended to be a poor medium of exchange and an even worse unit of account.
In order for something to perform well as a medium of exchange or unit of account it must remain
relatively stable against other goods and services, because money is ultimately a good that other goods
are denominated in. If the price of money as a good is too variable then it becomes less useful as a
denominator of other goods. \\

\subsection{Stablecoins}

\noindent A stablecoin is a cryptocurrency designed for price stability, such that it can function as
both a medium of exchange and unit of account. It should ideally be as effective for making payments
as fiat currencies like the US Dollar, but still retain the desirable characteristics of Bitcoin;
transaction immutability, censorship resistance and decentralisation. \\

\noindent Cryptocurrencies are in these ways a far better form of money, but have been significantly hindered
in their adoption by the fact that as decentralised systems, they have had relatively inflexible internal
monetary policies which have fostered volatility and deflation. Hence stability continues to be one of the
most valuable and yet the most elusive characteristics.
The fact that we have yet to achieve stability in cryptocurrencies without resorting to extreme
centralisation should by no means be taken as evidence that this problem is insurmountable.
The reality is that the technology to create alternative monetary policies within
cryptoeconomic systems has only existed for a few years. Clearly, significant research into stable monetary
frameworks for cryptocurrencies is required. \\

\subsection{Havven}
\noindent The mechanism Havven uses for maintaining its peg relies on two linked tokens and a
complex of incentives for stability:

\paragraph{Nomin} The stablecoin itself, whose supply floats. Its price measured in fiat currency should be relatively stable.
Other than price stability, the system should also encourage some adequate level of liquidity for nomins
to act as a useful medium of exchange.

\paragraph{Curit} The collateral token, whose supply is static.
The capitalisation of the curits in the market reflects the system's aggregate value, and the reserve
which backs the stablecoin. Thus, users who hold curits take on the role of maintaining the peg. \\

\noindent Each holder of curits is granted the right to issue a value of nomins in proportion to the USD value
of the curits they hold and are willing to place into escrow. If the user wishes to redeem their escrowed curits, they must
present the system with nomins in order to free their curits and trade them again.
The holders of this token provide both collateral and liquidity, and in so doing assume some
level of risk. To compensate this risk, nomin-issuers will be rewarded with fees the system levies
automatically as part of its normal operation.

In this manner, the system incentivises the issuance and destruction of nomins so that the value of
the nomin pool expands and contracts in proportion with the total value of curits backing them.
If the curit price changes, then the volume of the token pool changes with it.
On the other hand, if the nomin price changes exogenously, then the system is designed to provide
incentives for actors to counteract that change. \\

\subsection{Rationale}

It is clear that the introduction of a new cryptocurrency, in isolation, offers no additional value given
the existing and established alternatives such as Bitcoin and Ethereum. Havven thus seeks to derive value
from the addition of \textbf{stability} to its inherited properties as a modern cryptocurrency.
It is designed to provide a practical medium of exchange, without compromising the benefits that decentralisation
offers in order to substantially improve the technology of money.

There are many applications which Bitcoin's inherently deflationary monetary policy and
volatility presently make impossible. So any token which is able to demonstrate an increment
in utility on these fronts over both fiat and cryptocurrencies will significantly
enhance the uptake of cryptocurrency.

\noindent  In his discussion of Hayek money~\cite{ametrano2016hayek}, Ametrano correctly makes the point that
Bitcoin serves the purpose of crypto-gold much better than it does crypto-unit-of-account due to its volatility
and constrained supply. By contrast, governments -- which mint their own currencies -- can and do execute
discretionary stabilisation policies to manipulate the circulating supply. This kind of powerful lever is not
available to Bitcoin and other supply-constrained currencies of its type, but a similar system whose monetary
policy is algorithmically countercyclical rather than deflationary could inherit the desirable characteristics
of both monetary paradigms. It should be possible, by automatic means, to incentivise the issuance and
destruction of tokens according to demand. In this way, users of such a currency would be allowed to
capitalise it while the system automatically seeks to expand and contract the money supply as its backing
reserve fluctuates in value. By this mechanism we might produce a more perfect currency where supply floats
with necessity, but which is not prone to debasement and other issues commonly associated with
inflationary or deflationary forms of money. Ideally, we also seek to remove some of the distortions created
by traditional monetary policy, which, when it is expansionary, shrinks the purchasing power in every account
which is not a direct beneficiary of that policy.\\

\noindent The Havven stablecoin system is akin to representative money in the sense that the fungible nomin
tokens represent some value held in reserve. We define the curit to be the token of backing value as this is
both the start and end point of using the Havven mechanism; curits develop intrinsic value given their ability
to maintain stability (through nomins) with an external denomination. Hence, nomins have no intrinsic value
because we define curits as carrying the value associated with being able to provide a functioning stable
medium of exchange. \\

\noindent Havven however is not representative money as we have traditionally known it. Historical instantiations,
such as the gold standard which allowed anyone to claim against the reserve, caused exacerbations in times of
economic turmoil. Given that it does not need to act as the primary currency in the market, Havven is relieved
of any pressure to respond and correct for macroeconomic market issues. We leave such manipulations of the money
supply to the whims of central banks, for good or ill. Thus Havven is at its simplest a bridge between fiat and
cryptocurrency, a hybrid of the two technologies and thus for numerous use cases superior to both. But it bears
repeating that whatever monetary policy is applied to the external denomination will flow through to the system.
For example, if the USD is significantly devalued through inflation, so too will the nomin. In this scenario,
the value of curits against the USD will increase and more nomins will be able to be issued against that value,
so long as they are denominated in USD. \\

\noindent The Havven system is designed such that the nomin is both denominated in and
mapped to an external store of value. Throughout this paper we use USD as the reference, however this could
be any external and appropriately fungible asset, such as a commodity or fiat currency. Note that denominations
in other cryptocurrencies are not necessary as these already benefit from the features Havven is implementing
for the external denominator. \\

\pagebreak
